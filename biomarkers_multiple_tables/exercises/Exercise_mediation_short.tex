\documentclass[11pt]{article}
\usepackage[a4paper,top=3cm,bottom=3cm,left=2cm,right=2cm]{geometry}
%\usepackage[authoryear,round]{natbib}
\usepackage{hyperref}
\usepackage[pdftex]{color,graphicx,epsfig}
\DeclareGraphicsRule{.pdftex}{pdf}{.pdftex}{}
\usepackage{amssymb,amsmath}
\usepackage{enumerate}
\usepackage[spanish]{babel}
\usepackage[ansinew]{inputenc}



\begin{document}


\title{\bf Mediation analysis}
\date{}


\maketitle


%%%%%%%%%%%%%%%%%%%%%%%%%%%%%%%%%%%%%%%%%%%%%%%%%%%%%%%%%%%%%%%%%

%%%%%%%%%%%%%%%%%%%%%%%%%%%%%%%%%%%%%%%%%%%%%%%%%%%%%%%%%%%%%%%%%%%

\noindent {\bf TASK:} Load file {\tt diet2.Rdata} into R by using {\tt load} command. The function loads an object called {\tt diet} that is a data.frame including information about 4861 individulas and several variables including diseases and confounding variables (columns 1-16), nutrients (columns 17-26) and food consumption (columns 27-48). 

\bigskip

\noindent Perform mediation analysis considering colesterol as a mediator of the association observed between consumption of processed meat (variable {\tt gra\_conveniencefood}) and colorectal cancer (variable {\tt casoc}). NOTE: adjust the models by {\tt sexo}, {\tt t\_energy}, {\tt BMIhoy} and {\tt edad}.

\end{document}


%%%%%%%%%%%%%%%%%%%%%%%%%%%%%%%%%%%%%%%%%%%%%%%%%%%%%%%%%%%%%%%%%%%%%%%%%%%%%%%%%%%%%%%%%%%%%%%%%%%%%%%%%%%%%%%%%%%
