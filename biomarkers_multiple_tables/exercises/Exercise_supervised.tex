\documentclass[11pt]{article}
\usepackage[a4paper,top=3cm,bottom=3cm,left=2cm,right=2cm]{geometry}
%\usepackage[authoryear,round]{natbib}
\usepackage{hyperref}
\usepackage[pdftex]{color,graphicx,epsfig}
\DeclareGraphicsRule{.pdftex}{pdf}{.pdftex}{}
\usepackage{amssymb,amsmath}
\usepackage{enumerate}
\usepackage[spanish]{babel}
\usepackage[ansinew]{inputenc}



\begin{document}


\title{\bf Supervised methods}
\date{}


\maketitle


%%%%%%%%%%%%%%%%%%%%%%%%%%%%%%%%%%%%%%%%%%%%%%%%%%%%%%%%%%%%%%%%%

%%%%%%%%%%%%%%%%%%%%%%%%%%%%%%%%%%%%%%%%%%%%%%%%%%%%%%%%%%%%%%%%%%%

\noindent {\bf TASK 1 - Supervised methods:} File {\tt diet.dta} is a Stata database including information about several diseases and confounding variables (columns 1-16), nutrients (columns 17-26) and food consumption (columns 27-48). 

 
\begin{enumerate}
 \item Load data into R and save the information in an object called {\tt diet}.
 \item Create a train database selecting 4000 samples randomly and a test database with the rest by executing:
 
 \begin{verbatim}

 library(readstata13)
 diet <- read.dta13("data_exercises/diet.dta")

 vars <- !names(diet)%in%c("id", "casoc", "casom", 
                          "casop", "casoe")
 diet <- diet[ , vars]
 diet <- diet[complete.cases(diet),]
 diet$tipocancer <- droplevels(diet$tipocancer)

 
 set.seed(12345) 
 sel <- sample(1:nrow(diet), 4000)
 train <- diet[sel, ]
 test <- diet[-sel,]
 \end{verbatim}
 
 \item Perform a variable selection using Random Forest and Linear Discriminat methods using train dataset to predict the different types
 of cancer (variable {\tt tipocancer}).
 \item Evaluate model performance using the test dataset. 
\end{enumerate}
 

\end{document}


%%%%%%%%%%%%%%%%%%%%%%%%%%%%%%%%%%%%%%%%%%%%%%%%%%%%%%%%%%%%%%%%%%%%%%%%%%%%%%%%%%%%%%%%%%%%%%%%%%%%%%%%%%%%%%%%%%%
